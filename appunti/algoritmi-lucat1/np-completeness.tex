\documentclass{article}
\usepackage[utf8]{inputenc}
\usepackage{amsmath,amssymb,amsthm}
\usepackage{enumerate}
\usepackage[ruled]{algorithm2e}
\usepackage{float}

\DontPrintSemicolon
\renewcommand{\thealgocf}{} % don't print algorithm numbers
\hfuzz=100.0pt  % ignore paragraph lengths warnings

\title{\textbf{Teoria della NP completezza}}
\author{Luca Tagliavini}
\date{March 17-.., 2021}

\begin{document}

\maketitle

\section{Introduzione}

I problemi risolvibili tramite algoritmi hanno un costo associato come sappaimo.
I problemi NP-completi cercano di capire se un qualcosa e' computabile, entro
un dato limite. I risultati di tali ragionamenti saranno dunque valori booleani.

Ecco alcuni esempi di problemi NP-completi:
\begin{enumerate}
  \item \textbf{Commesso viaggiatore}: dato un grafo completo ed un valore $k$,
    decidere se esiste un cammino semplice che copre tutti i vertici di peso inferiore a $k$.
  \item \textbf{Bin packing problem}: dati $n$ oggetti ognungo con un peso $P[n]$
    e $k$ contenitori di capacita' $c$, decidere se e' possibile distribuire tutti
    gli $n$ oggetti nei $k$ contenitori senza eccedere la capacita' $c$.
  \item \textbf{Sudoku}: data una matrice di dimensione $n^2 \times n^2$(sudoku
    classico $9 \times 9$ organizzata in blocchi di dimensioni $n \times n$ (sudoku
    classico $3 \times 3$) con alcune celle gia' riempite, sia possibile riempire
    le restanti in modo che ogni riga, ogni colonna e ogni blocco $n \times n$
    contenga tutti i numrei compresi tra $1 e n^2$.
\end{enumerate}

\subsection{Formalizzazione}

Consideriamo un problema $K$ come una relazione tra $Q \subseteq I \times S$ dove
\begin{itemize}
  \item $I$ e' l'insieme delle \emph{istanze d'ingresso}.
  \item $S$ e' l'insieme delle \emph{soluzioni}.
\end{itemize}
Immaginiamo $Q$ come un predicato che preso in input il dato $x \in I$ e la
soluzione $s \in S$ restituisce:
\begin{enumerate}
  \item $1 \iff (x,s) \in Q$ ($s$ soluzione del problema $Q$ sull'istanza $x$)
  \item $0$ altrimenti ($s$ \textbf{non} e' soluzione del problema $Q$ sull'istanza $x$)
\end{enumerate}

\subsubsection{Tipologie di problemi}

Esistono problemi di vario tipo, come ricerca (trovare un elemento in un array)
o di ottimizzazione (trovare il cammino minimo) ma sappiamo da prima che ogni
problema di un qualunque tipo ha un suo problema corrispondente del tipo
\textbf{decisionale} che restituisce un booleano in base alla \emph{trattabilita'}
del problema.

\subsection{Classi di complessita'}

Data una funzione $f(n)$, chiamiamo $TIME(f(n))$ o $SPACE(f(n))$ l'insieme di tutti
i problemi che hanno una soluzione algoritmica di costo computazionale $O(f(n))$.

\subsubsection{Classe polinomiale}

La classe $P$ dei problemi risolvibili in \textbf{tempo} \emph{polinomiale} nella
dimensione $n$ dell'istanza d'ingresso:
\begin{align*}
  P = \cup^{\infty}_{c=0} TIME(n^c)
\end{align*}

La classe $PSPACE$ dei problemi risolvibili in \textbf{spazio} \emph{polinomiale}
nella dimensione $n$ dell'istanza d'ingresso:
\begin{align*}
  PSPACE = \cup^{\infty}_{c=0} SPACE(n^c)
\end{align*}

\subsubsection{Classe esponenziale}

La classe $EXPTIME$ dei problemi risolvibini il \textbf{tempo} \emph{esponenziale}
nella dimensione $n$ dell'istanza d'ingresso:
\begin{align*}
  EXPTIME = \cup^{\infty}_{c=0} TIME(2^{n^c})
\end{align*}

\subsubsection{Ordinamento delle classi}

E' abbastanza facile inutire come $P \subset EXPTIME$, in quanto:
\begin{enumerate}
  \item Entambe misurano il tempo.
  \item Un problema con tempo polinomiale sara' sempre di complessita'
    inferiore rispetto a uno di tempo esponenziale.
\end{enumerate}

\noindent Capiamo oltretutto che $P \subseteq PSPACE$ in quanto un algoritmo
che richiede un tempo polinomiale al piu' ad accedere ad un numero polinomiale
di locazioni di verse. \\
Si puo' oltretutto capire che $PSPACE \subseteq EXPTIME$, in quanto
una memoria di $n^c$ locazioni puo' essere al piu' essere in $2^{n^c}$ stati,
che sono sempre inferiori ad un numero esponenziale.

\subsection{Esempio: problema SAT}

Il problema SAT, ovvero della soddisfacibilita' di una espressione booleana, che
contiene solo operazioni di $\wedge, \vee, \overline{\cdot}$. Anche espressioni
booleane piu' complesse possono sempre essere ridotte a \emph{forme normali congiuntive},
ovvero forme che contegono piu' clausole ($\vee$ tra letterali, possibilmente negati)
a loro volta unite da una serie di $\wedge$.

Una possibile soluzione al problema e' provare ogni coppia di soluzioni, dando
ogni possibile valore a ogni $n$ variabile. Prenderemo dunque coppie $\overline{c}
\in \mathbb{B}^n$ che ha dimensione $2^n$, il che rende il problema appartenente
alla famiglia $PSPACE$.

\subsection{Certificare vs Verificare}

Quando ci viene chiesto di \emph{verificare} se una determinata formula booleana e'
soddisfacibile per un numero $n$ di variabili, ci basta capire se esiste almeno
una coppia di $n$ variabili tale che l'espressione vale. \\
Tuttavia, e' desiderabile non solo capire se un problema e' risolvibile, ma bensi'
\emph{certificare} restituendo un valore che accerta che tale problema ammette la soluzione data.

\subsection{La cateogirna NP}

La cateogira $NP$ e' quella categoria di problemi che ammettono \emph{certificati
vertificabili} in tempo polinomiale.

\subsection{Non determinismo}

Gli algoritmi visti in precedenza si comportano in maniera \emph{deterministica},
le loro mosse sono prevedibili e ripercorribili in modo sicuro e oggettivo. Immaginiamo
ora di ricevere un aiuto esterno, che ci consente di trovare una soluzione conveniente
per un determinato problema, e fa proseguire il nostro algoritmo nella direzione giusta.
Questa tecnica di soluzione e' detta \textbf{non deterministica}, poiche' non
ha uno svolgimento prevedibile.

L'albero di tute le scelte \emph{non deterministiche} fatte viene rappresentato
come un albero che ha nelle foglie valori compresi in $\mathbb{B}$, dove $1$ indica
il successo dell'algoritmo, mentre lo $0$ indica una strada seguita ma non valida.
Se esiste almeno una foglia con valore $1$ l'intero calcolo e' fallito e' corretto,
e si restituisce dunque $1$, altrimenti $0$.

\subsection{Classe polinomiale non deterministica}

Data una funzione $f(n)$, chiamiamo $NTIME(f(n))$ l'insieme dei problemi decisionali
che possono essere risolti da un algoritmo non deterministico in tempo $O(f(n))$.
La classe $NP$ e' quella dei problemi risolvibili in tempo polinomiale non
deterministico nella dimensione $n$ dell'istanza d'ingresso:
\begin{align*}
  NP = \cup^{\infty}_{c=0} NTIME(n^c)
\end{align*}

\end{document}
