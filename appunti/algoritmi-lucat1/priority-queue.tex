\documentclass{article}
\usepackage[utf8]{inputenc}
\usepackage{amsmath,amssymb,amsthm}
\usepackage{listings}
\usepackage{enumerate}

\title{\textbf{Code con priorita'}}
\author{Luca Tagliavini}
\date{April 8-12, 2021}

\begin{document}

\maketitle
\tableofcontents
\pagebreak

\subsection{Coda con priorita'}

Struttura che mantiene il minimo (o massimo) in un insieme dinamico di chiavi
su cui e' definita una relazione d'ordine totale. Una coda con priorita' e' un 
insieme di $n$ elementi di tipo $elem$ cui sono associate chiavi.
\begin{itemize}
  \item $k_1 \Rightarrow elem_1$
  \item $k_2 \Rightarrow elem_2$
  \item $k_3 \Rightarrow elem_3$
  \item $\ldots$
\end{itemize}

Un esempio puo' essere la sincronizzazione della banda dove c'e' un continuo
scambio di pacchetti tra client e server, e alcuni pacchetti hanno giustamente
priorita' piu' alta e vengono dunque scambiati prima di altri (i.e. uno stream
audio del presentatore della lezione vs un messaggio in chat).
I pacchetti in ingresso possono dunque essere mantenuti in una coda con priorita'
per gestire prima quelli con priorita' maggiore. Una cosa analoga puo' essere fatta
lato server dove i pacchetti vengono distribuiti ai vari client.

\subsubsection{Operazioni disponbili}

\begin{itemize}
  \item \emph{find\_min()} $\rightarrow$ \emph{elem}: restituisce l'elemento associato
    alla chiave minima.
  \item \emph{insert(elem e, chiave k)} $\rightarrow$ \emph{void}: inserisce un nuovo
    elemento $e$ con associata la chiave $k$.
  \item \emph{delete(elem e)} $\rightarrow$ \emph{void}: rimuove un elemento dalla coda
    al quale assumiamo di avere si ha accesso diretto.
  \item \emph{delete\_min()} $\rightarrow$ \emph{void}: rimuove l'elemento associato alla
    chiave minima.
  \item \emph{increase\_key(elem e, chiave k)} $\rightarrow$ \emph{void}: aumenta la
    priorita' dell'elemento $e$ della quantita' $k$.
  \item \emph{decrease\_key(elem e, chiave k)} $\rightarrow$ \emph{void}: decrementa la
    chiave dell'elemento $e$ eella quantita' $d$
\end{itemize}

La struttura ammette metodi e funzionalita' molto simili quindi ci ispireremo
fortemente alla sua costruzione. Tuttavia applicheremo alcuni cambiamenti:
\begin{itemize}
  \item Non avremo piu' un albero binario, ma ogni nodo potra' avere un numero
    arbitrario di figli fino ad un numero massimo $d$. Chiameremo questa
    struttura $d$-heap.
  \item Implementeremo le operazioni \emph{delete, increase\_key, decrease\_key}
    che non avevamo implementato ai tempi dell'heap sort poiche' non sfruttate.
\end{itemize}

NOTA: \emph{increase\_key} tornera' utile quando studieremo l'agoritmo di Dijkstra.

\subsection{$d$-heap}

Estende il concetto di min/max-heap gia' visto, con la differenza che un heap
classico e' binario mentre invece un $d$-heap sara' $d$-ario.
Ha le seguenti proprieta':
\begin{itemize}
  \item Un $d$-heap di $h$ altezza e' un albero perfetto almeno fino alla
    profondita' $h-1$. Le foglie (nel livello $h$ vengon accatastate a sinistra).
  \item Ogni nodo $v$ contiene una $chiave(v)$ e un $elem(v)$. L'insieme delle
    chiavi e' un insieme ordinato (i.e. $1,2,3,\ldots,10$)
  \item Ogni figlio (diverso dalla radice) ha la chiave \emph{non inferiore}
    ($\geq$) a quella del padre.
  \item Un $d$-heap con $n$ nodi ha profondita' $O(\log_d n)$. \\
    Lo si prova con le serie geometriche come visto per le normali heap, vdi slide.
\end{itemize}

Anche i $d$-heap possono essere memorizzati tramite un array (index $1-n$), il
cui processo e' descritto nelle slide e ci offre un accesso diretto agli
elementi in posizioni arbitrarie in modo da abbassare il costo di alcuni metodi
(altrimenti logaritmici) a costanti.

\subsubsection{Funzioni ausiliarie: \emph{muovi}\_\emph{basso} }

\begin{lstlisting}
algorithm muovi_basso(Node v)
  while true do
    if has_no_child(v) then
      break
    else
      u := min_child(v)
      if chiave(u) < chiave(v) then
        swap(u, v)
        v = u
      else
        break
    endif
  endwhile
endalgoritmh
\end{lstlisting}

Costo: $O(dh)$

Sposta il nodo $v$ in basso nella heap (tra i suoi figli) fino ad una posizione
in cui rende l'albero un $d$-heap, ossia posiziona il nodo $v$ scambiandolo con
un figlio in modo che esso sia il minore tra il sottoalbero preso in considerazione.

\subsubsection{Funzioni ausiliarie: \emph{muovi}\_\emph{alto} }

\begin{lstlisting}
algorithm muovi_alto(Node v)
  while v != root(T) and chiave(v) < chiave(parent(v)) do
    swap(v, parent(v))
    v = parent(v)
  endwhile
endalgoritmh
\end{lstlisting}

Costo: $O(d)$

Sposta il nodo $v$ in alto fino a quando o esso non e' la radice (ci fermiamo
perche' la chiamata \emph{parent} fallirebbe) o \textbf{ha una chiave $\geq$ di
quella del padre}

\subsubsection{Costi degli altri metodi della $d$-heap}

Si noti che $d = \log_d n$ per le proprieta' della $d$-heap.

\begin{itemize}
  \item \emph{find\_min()} $\Rightarrow$ $O(1)$ in quanto il minimo e' il
    nodo radice
  \item \emph{insert(elem e, chiave k)} $\Rightarrow$ $O(\log_d n)$ in quanto
    aggiunge l'elemento il piu' possibile a destra nell'ultimo livello e necessita
    di una chiamata di \emph{muovi\_alto}
  \item \emph{delete(elem e)} $\Rightarrow$ $O(d \log_d n)$ in quanto si scambia
    la foglia piu' a destra dell'ultimo livello con il nodo scambiato e poi si
    esegue \emph{muovi\_alto} o \emph{muovi\_basso} in base alla situazione, tra
    i quali il peggiore ha costo $O(d \log_d n)$
  \item \emph{delete\_min()} $\Rightarrow$ $O(d \log_d n)$ in quanto chiama \emph{delete}
  \item \emph{increase\_key(elem e, chiave k)} $\Rightarrow$ $O(\log_d n)$ in
    quanto aumenta il valore e chiama \emph{muovi\_alto} per riordinare l'albero
    dopo l'incremento
  \item \emph{decrease\_key(elem e, chiave k)} $\Rightarrow$ $O(d \log_d n)$ in
    quanto chiama \emph{muovi\_basso} per ribilanciare la $d$-heap
\end{itemize}

\end{document}
