\documentclass{article}
\usepackage[utf8]{inputenc}
\usepackage{amsmath,amssymb,amsthm}
\usepackage{listings}
\usepackage{enumerate}

\title{\textbf{Selezione}}
\author{Luca Tagliavini}
\date{April 7-8, 2021}

\begin{document}

\maketitle
\tableofcontents
\pagebreak

\subsection{Il problema della selezione}

Il problema della \emph{selezione del k-esimo} e' il problema che rappresenta
l'ordinamento e la selezione dell'elemento $k$ dopo aver ordinato l'array in cui
esso e' contenuto. La domanda che ci facciamo e', \emph{si ha un modo migliore per
selezionare l'elemento anzi che ordinare l'intero array}?

Avremo dunque un array $A[1..n]$ di valori \emph{distinti}, un valore $1 \leq
k \leq n$ e dovremo trovare l'elemento che e' maggiore esattamente di $k-1$
elementi. Esempio trovare \emph{il mediano}: valore che occuperebbe la posizione
$n/2$ se l'array fosse ordinato.

\subsubsection{Caso particolare: selezione del minimo}

Ricerca del $k$-esimo con $k = 1$:

\begin{lstlisting}
algorithm minimum(T[1..n] A) -> T
  min := A[1]
  for i := 2 to n do
    if A[i] < min then
      min = A[i]
    endif
  endfor
  return min
endalgorithm
\end{lstlisting}

Costo: $O(n-1) = \Theta(n)$

\subsubsection{Caso particolare: selezione del \emph{secondo} minimo}

Ricerca del $k$-esimo con $k = 2$:

\begin{lstlisting}
algorithm second_minimum(T[1..n] A) -> T
  min1 := A[1]
  min2 := A[2]
  for i := 3 to n do
    if A[i] < min2 then
      min2 = A[i]
      if min2 < min1 then
        swap(min1, min2)
      endif
    endif
  endfor
  return min
endalgorithm
\end{lstlisting}

Costo: $O(2(n-2)+1)$

\subsection{Generalizzazione: ricerca (non ottimale) del $k$-esimo}

Questo algoritmo si ispira fortemente al Selection Sort, in quanto cerca $k$
volte il minimo e lo pone in cima all'array, per poi ritornare $A[k]$ che per
quello che abbiamo detto poc'anzi sara' il $k$-esimo valore minimo

\begin{lstlisting}
algorithm k_minimum(T[1..n] A, int k) -> T
  for i := 1 to k do
    minIndex := i
    minValue := A[i]
    for j := i+1 to n do
      if A[i] < minValue then
        minIndex := j
        minValue := A[j]
      endif
    endfor
    swap(A[i], A[minIndex])
  endfor
  return A[k]
endalgorithm
\end{lstlisting}

Costo: $O(kn)$

\subsection{Soluzione al problema del $k$-esimo con heap}

Si puo' avere una soluzione simile all'algoritmo dell'Heap Sort, che funziona
tramite una struttura arborescente Heap che posizione in testa all'array il
valore minore (in tempo lineare) e ne rimuove via via $k$ volte il piu' piccolo
valore per ottenere il minimo di nostro interesse (con tempositiche dell'ordine
$\log_2 n$). Eccone una implementazione:

\begin{lstlisting}
algorithm k_minimum(T[1..n] A, int k) -> T
  min_heapify(A)  // O(n)
  for i := 1 to k-1 do
    delete_min(A) // O(log n)
  endfor
  return find_min(A) // O(1)
endalgorithm
\end{lstlisting}

Costo: $O(n + k \log_2 n)$

Questa versione ispirata all'Heap Sort e' migliore di quella precedente basata
sul Selection Sort in quanto, per valori di $k$ molto grandi (vicini a $n$)
questo e' migliore.

In particolare se interessati a un $k \leq O(\frac{n} {\log_2 n})) = O(\frac{n}
{\log_2 n})$ avremo un tempo $O(n + \frac{n}{\log_2 n} \log_2 n) = O(2n) = O(n)$.
Ha chiaramente un costo migliore dell'altro (che sarebbe stato $O(\frac{n^2}
{\log_2 n})$.

Analogamente, per casi scomodi come la ricerca del mediano dove si ha $k = n/2$.
Facendo i conti avremmo un costo nell'ordine di $O(n \log_2 n)$, stesso costo di
un algoritmo di ordinamento completo.

\subsection{Metodo ancora migliore}

I spirandoci al Quick Sort possiamo tuttavia trovare una soluzione divide-et-impera
che ci portera' ad avere performance migliori delle soluzioni precedenti:

\begin{lstlisting}
algorithm select1(T[1..n] A, int k) -> T
  x := un elemento in A
  A1 := { y in A | y < x }
  A2 := { y in A | y = x }
  A3 := { y in A | y > x }
  quicksort(A1);
  quicksort(A3);
  ritorna merge(A1, A2, A3)[k];
endalgorithm
\end{lstlisting}

Ora possiamo osservare che, visto che il nostro obbiettivo non e' ordinare
l'intero array ma sono la parte che conterra' il $k$-esimo, possiamo vedere
dove andra' a cadere $k$ guardando le grandezze delle partizioni e ordinderemo
solo codeste partizioni. In questo modo avremo un costo vantaggioso.
Eccone una implementazione:

\begin{lstlisting}
algorithm quick_select(T[1..n] A, int k) -> T
  x := un elemento in A
  A1 := { y in A | y < x }
  A2 := { y in A | y = x }
  A3 := { y in A | y > x }
  if k <= len(A1) then
    return quick_select(A1, k)
  elif k <= len(A1) + len(A2) then
    return x
  else
    return quick_select(A3, k - len(A1) - len(A2))
  endif
endalgorithm
\end{lstlisting}

Calcolo del costo:
\begin{itemize}
  \item caso ottimo (ogni chiamata dimezziamo): $T(n) = T(n/2) + n = \Theta(n)$ \\
    caso 3 del Master Theorem. (NOTA: $+ n$ e' dovuto alla divisione in 3 sottoarray)
  \item caso pessimo (rimuoviamo sempre solo il pivot): $T(n) = T(n-1) + n = \Theta(n^2)$ \\
    dimostrabile come nel Quick Sort
  \item caso medio: Ad ogni iterazione elimino $len(A2) + min(len(A1), len(A3)$
  elementi, insomma assumo sempre che andro' a cercare nel sottoinsieme piu' grande.
  Il numero di elementi \emph{scartati} sara' dunque $1 \leq \cdot \leq n/2$. \\
  La propobabilita' di ricadere su un segmento di lunghezza $i$ e' di
    $\approx \frac{1}{n/2} = 2/n$ per $i=n/2, n/(2+1), \ldots, n-1$
  \begin{align*}
    T(n) = \begin{cases}
      1 &\text{se } n = 1 \\
      T(n') + n &\text{con } \frac{n}{2} \leq n' < n
    \end{cases}
  \end{align*}
  Tramite la tecnica della sostituzione proviamo il seguente teorema:
  Il costo $T(n) \leq 4n$, quindi $T(n) = O(n)$.
\end{itemize}

\end{document}
